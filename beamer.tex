% 4:3 (default) aspect ratio
%\documentclass[t, xcolor=table]{beamer}
% 16:9 aspect ratio
\documentclass[t, xcolor=table, aspectratio=169]{beamer}

% Basic packages which are recommended for the KUL beamer theme
\usepackage[utf8]{inputenc}
\usepackage[T1]{fontenc}
\usepackage[UKenglish]{datetime}

%-------------------------%
% Add extra packages here %
%-------------------------%

% Apply the KUL beamer theme
\usetheme{KUL}
% use the include below to change to the KU Leuven Geel logo
%\usetheme[logo=geel]{KUL}

% Optional change of minted highlight style
% \usemintedstyle{friendly}

% Set date and time format
\newdateformat{SetNewDate}{%
    \twodigit{\THEDAY}/\twodigit{\THEMONTH}/\THEYEAR
}
\SetNewDate

% Setup the title page
\title{KU Leuven\\\indent \LaTeX{} beamer template}
\subtitle{Version \KULversion}
\institute{Campus Geel, Faculty of industrial engineering sciences}
\author{John Doe}
\date{\today}

%---------------%
% Main document %
%---------------%
\begin{document}
    % Title page
    % ----------
    \begin{frame}[KUL-title]
        \titlepage
    \end{frame}

    % Outline
    % -------
    \begin{frame}[KUL-standard-notoc]
        \frametitle{Outline}
        \vspace{-1em}
        \tableofcontentsTOP %aligned at the top of the page
        %\tableofcontents %aligned at the center of the page
    \end{frame}

    % Outline with multiple columns
    \begin{frame}[KUL-standard-notoc]
        \frametitle{Outline multiple columns}
        \vspace{1em}
        \begin{columns}[onlytextwidth,T]
            \begin{column}{0.5\textwidth}
                \tableofcontentsTOP[sections={1}]
            \end{column}
            \begin{column}{0.5\textwidth}
                \tableofcontentsTOP[sections={2-3}]
            \end{column}
        \end{columns}
        \vspace{1em}
        Using the \texttt{column} environment, you can split the ToC over multiple columns.
    \end{frame}

    % Simple markup
    % -------------
    \begin{frame}[KUL-header]
        \frametitle{Simple markup}
        \framesubtitle{(using \LaTeX{} obviously)}
    \end{frame}

    % Formatting
    \begin{frame}[KUL-standard]
        \frametitle{Formatting}
        \textbf{Bold text}
        \par\ \par
        \textit{Italicised text}
        \par\ \par
        \textul{Underlined text}
        \par\ \par
        \textst{Striked through text}
        \par\ \par
        \textsc{Small caps text}
        \par\ \par
        \textcolor{red}{Coloured text}
    \end{frame}

    % Unordered lists
    \begin{frame}[KUL-standard]
        \frametitle{Unordered list}
        \begin{itemize}
            \item First level
            \begin{itemize}
                \item Second level
                \begin{itemize}
                    \item Third level
                    \begin{itemize}
                        \item Fourth level
                        \begin{itemize}
                            \item Fifth level
                        \end{itemize}
                    \end{itemize}
                \end{itemize}
            \end{itemize}
            \item ...
        \end{itemize}
    \end{frame}

    % Ordered lists
    \begin{frame}[KUL-standard]
        \frametitle{Ordered list}
        \begin{enumerate}
            \item First level
            \begin{enumerate}
                \item Second level
                \begin{enumerate}
                    \item Third level
                    \begin{enumerate}
                        \item Fourth level
                        \begin{enumerate}
                            \item Fifth level
                        \end{enumerate}
                    \end{enumerate}
                \end{enumerate}
            \end{enumerate}
            \item ...
        \end{enumerate}
    \end{frame}

    % Hyperlinks
    \begin{frame}[KUL-standard]
        \frametitle{Hyperlinks}
        \begin{itemize}
            \item A hyperlink to some interesting page at
                  \href{https://en.wikipedia.org/wiki/LaTeX}{wikipedia}.
            \vspace{2em} % 2 lines of spacing
            \item The same, but with the full URL:
                  \url{https://en.wikipedia.org/wiki/LaTeX}
            \vspace{2em} % 2 lines of spacing
            \item And a special mailto command:
                  \mailto{john.doe@example.com}
        \end{itemize}
    \end{frame}

    % Images
    \begin{frame}[KUL-standard]
        \frametitle{Images}

        % c = vertically align center, T = vertically align top,
        % b = vertically align bottom
        \begin{columns}[onlytextwidth,c]
            % First column contains some bullet points
            \begin{column}{0.5\textwidth}
                \begin{itemize}
                    \item Some interesting info about the image...
                    \vspace{2em}
                    \item More info about the image...
                    \vspace{2em}
                    \item Something seems familiar about the image though...
                \end{itemize}
            \end{column}
            % Second column contains the image
            \begin{column}{0.5\textwidth}
                % Note that \textwidth here, is the width of the column, not
                % of the entire slide (like before with the column)
                \includegraphics[width=\textwidth]{\KULlogopath}
            \end{column}
        \end{columns}
    \end{frame}

    % Tables
    % ------
    \begin{frame}[KUL-header]
        \frametitle{Tables}
    \end{frame}

    % Basic tables
    \begin{frame}[KUL-standard]
        \frametitle{Basic tables}

        \begin{itemize}
            \item Using a header row
        \end{itemize}

        \begin{KULtable}{\textwidth}{|L|C|R|}
            \hline
            \KULheadrow{Left aligned, Center aligned, Right aligned}
            \\ \hline
            a & b & c
            \\ \hline
            d & e & f
            \\ \hline
            g & h & i
            \\ \hline
        \end{KULtable}

        \begin{itemize}
            \item Using a header column
        \end{itemize}

        \begin{KULtable}{\textwidth}{|L|C|C|C|}
            \hline
            \KULheadcol{First column} & a & b & c
            \\ \hline
            \KULheadcol{Second column} & d & e & f
            \\ \hline
            \KULheadcol{Third column} & g & h & i
        \end{KULtable}
    \end{frame}

    % Combining columns and rows
    \begin{frame}[KUL-standard]
        \frametitle{Combining columns and rows}

        % Note that, since the rows are coloured, a 'multirow' statement needs
        % to come at the end of a column using a negative amount of rows.
        \begin{KULtable}{\textwidth}{|L|C|C|}
            \hline
            \KULheadrow{Groups, Results A, Results B}
            \\ \hline

            & 8 & 10
            \\ \cline{2-3}
            \multirow{-2}{*}{\cellcolor{KUL-BLUE-TABLE1}Group 1} &
                \textbf{\color{red}5} & 7
            \\ \hline

            & 7 & \textbf{\color{red}3}
            \\ \cline{2-3}
            \multirow{-2}{*}{\cellcolor{KUL-BLUE-TABLE2}Group 2} &
                9 & 6
            \\ \hline

            & \multicolumn{2}{c|}{7} %Note the small letter (i.e. C->c, L->l, ...)
            \\ \cline{2-3}
            \multirow{-2}{*}{\cellcolor{KUL-BLUE-TABLE1}Group 3} &
                \multicolumn{2}{c|}{\textbf{\color{red}2}}
            \\ \hline
            & 5 & 6
            \\ \cline{2-3}
            \multirow{-2}{*}{\cellcolor{KUL-BLUE-TABLE2}Group 4} &
                10 & 9
            \\ \hline
        \end{KULtable}
    \end{frame}

    % Vertically aligned tables
    \begin{frame}[KUL-standard]
        \frametitle{Vertically aligned tables}

        \begin{KULtable}{\textwidth}
        {|V{0.3\textwidth}|Z{0.3\textwidth}|W{0.3\textwidth}|}
            \hline
            \KULheadrow{Column 1, Column 2, Column 3}
            \\ \hline
                This is a long row with left alignment
                &
                This is a long row with center alignment
                &
                This is a long row with right alignment
            \\ \hline
                This uses \texttt{V}
                &
                This uses \texttt{Z}
                &
                This uses \texttt{W}
            \\ \hline
                Lorem ipsum dolor sit amet, consectetur adipiscing elit, sed do eiusmod tempor
                incididunt ut labore et dolore magna aliqua.
                &
                Lorem ipsum dolor sit amet, consectetur adipiscing elit, sed do eiusmod tempor
                incididunt ut labore et dolore magna aliqua.
                &
                Lorem ipsum dolor sit amet, consectetur adipiscing elit, sed do eiusmod tempor
                incididunt ut labore et dolore magna aliqua.
            \\ \hline
        \end{KULtable}
    \end{frame}

    % Other fancy stuff
    % -----------------
    \begin{frame}[KUL-header]
        \frametitle{Other fancy stuff}
    \end{frame}

    % Blocks
    \begin{frame}[KUL-standard]
        \frametitle{Blocks}
        \begin{block}{Block title}
            This is the format for a standard block
        \end{block}

        You can also change the width of blocks using the \texttt{varblock} environment.
        \begin{varblock}[0.5\textwidth]{Variable width block}
            Lorem ipsum dolor sit amet, consectetur adipiscing elit, sed do eiusmod tempor
            incididunt ut labore et dolore magna aliqua.
        \end{varblock}
    \end{frame}

    % Code blocks
    \begin{frame}[KUL-standard,fragile] % fragile is required for minted!!!
        \frametitle{Code blocks}
        The \texttt{minted} package is include by default, which allows you to highlight code blocks
        both inline (\mintinline{latex}{\LaTeX{} math: $a^2 + b^2 = c^2$}) and as a block:
        \begin{minted}[linenos]{latex}
\documentclass[12pt, a4paper]{article}
\begin{document}
    Hi there!
\end{document}
        \end{minted}
        \vfill
        \textbf{\color{red}Important}: If you use a \texttt{minted} code block, your slide has to be
        defined as \texttt{fragile}!
    \end{frame}

    % Code blocks
    \begin{frame}[KUL-standard-notoc,fragile]
        \frametitle{Code blocks}

        \newsavebox\BoxCodeInTable
        \begin{lrbox}{\BoxCodeInTable}
        \mintinline{latex}{\mintinline}
        \end{lrbox}

        \newsavebox\BoxCodeInTableLrBox
        \begin{lrbox}{\BoxCodeInTableLrBox}
        \mintinline{latex}{\begin{lrbox} ... \end{lrbox}}
        \end{lrbox}

        \newsavebox\BoxCodeInTableTwo
        \begin{lrbox}{\BoxCodeInTableTwo}
        \mintinline{latex}{\textit{italic text}}
        \end{lrbox}

        \begin{KULtable}{\textwidth}{|C|C|}
            \hline
            \KULheadrow{Code,Description}
            \\ \hline
            \usebox{\BoxCodeInTable} \newline
            \usebox{\BoxCodeInTableTwo} \newline
            \texttt{...}
            &
            You cannot use \usebox{\BoxCodeInTable} in certain environments such as tables. To fix
            this, wrap them inside a \usebox{\BoxCodeInTableLrBox}
            \\ \hline
        \end{KULtable}
    \end{frame}
\end{document}
